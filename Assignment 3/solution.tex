\documentclass[11pt]{article}
\usepackage{amsfonts}
\usepackage{amsmath}
\usepackage{latexsym}
\usepackage{hyperref}
\usepackage{pdfpages}
\usepackage{tikz}
\usepackage{wrapfig}
\usepackage{fancyvrb}
\usepackage{fontspec}
\usepackage{fancybox}
\usepackage{listings}
\usepackage{enumitem}
\usepackage{ducksay}
\usepackage{xcolor}
\usepackage{amssymb}
\usepackage{graphicx}
\usepackage{parskip}
\usepackage{subfig}
\graphicspath{ {./images/} }
\setlength{\oddsidemargin}{.25in}
\setlength{\evensidemargin}{.25in}
\setlength{\textwidth}{6in}
\setlength{\topmargin}{-0.4in}
\setlength{\textheight}{8.5in}
\setlength{\parindent}{0cm}

\def\squarebox#1{\hbox to #1{\hfill\vbox to #1{\vfill}}}
\def\qed{\hspace*{\fill}
        \vbox{\hrule\hbox{\vrule\squarebox{.667em}\vrule}\hrule}}
\newenvironment{solution}{\begin{trivlist}\item[]{\bf Solution:}}
                      {\qed \end{trivlist}}
\newenvironment{solsketch}{\begin{trivlist}\item[]{\bf Solution Sketch:}}
                      {\qed \end{trivlist}}
\newenvironment{proof}{\begin{trivlist}\item[]{\bf Proof:}}
                      {\qed \end{trivlist}}

\newtheorem{theorem}{Theorem}
\newtheorem{corollary}[theorem]{Corollary}
\newtheorem{lemma}[theorem]{Lemma}
\newtheorem{observation}[theorem]{Observation}
\newtheorem{remark}[theorem]{Remark}
\newtheorem{proposition}[theorem]{Proposition}
\newtheorem{definition}[theorem]{Definition}
\newtheorem{Assertion}[theorem]{Assertion}
\newtheorem{fact}[theorem]{Fact}
\newtheorem{hypothesis}[theorem]{Hypothesis}
%\newtheorem{observation}[theorem]{Observation}
%\newtheorem{proposition}[theorem]{Proposition}
\newtheorem{claim}[theorem]{Claim}
\newtheorem{assumption}[theorem]{Assumption}

%Put more macros here, as needed.
\newcommand{\al}{\alpha}
\newcommand{\Z}{\mathbb Z}
\newcommand{\jac}[2]{\left(\frac{#1}{#2}\right)}
\newcommand{\set}[1]{\{#1\}}
\newcommand{\evenSpace}{\vspace*{\stretch{1}}}
% Assignment header with the appropriate information
% 1st arg: Group member names
% 2nd arg: Assignment #
\newcommand{\header}[2]{
  \begin{center}
	\setlength\fboxsep{.3cm}
	\doublebox{
		\parbox{\dimexpr\linewidth-2\fboxsep-2\fboxrule} {
			#1 \\
			COSC 336 \\
			\today \par
			\centering{\huge{Assignment #2}}
		}}
\end{center}
}

\def\ppt{{\sf PPT}}
\def\poly{{\sf poly}}
\def\negl{{\sf negl}}
\def\owf{{\sf OWF}}
\def\owp{{\sf OWP}}
\def\tdp{{\sf TDP}}
\def\prg{{\sf PRG}}
\def\prf{{\sf PRF}}
\definecolor{variableColor}{HTML}{AA7700}
\definecolor{commentsColor}{rgb}{0.497495, 0.497587, 0.497464}
\definecolor{keywordsColor}{rgb}{0.00000, 0.000000, 1.500000}
\definecolor{stringColor}{rgb}{0.558215, 0.000000, 0.135316}
\lstset {
  backgroundcolor=\color{white},   % choose the background color; you must add \usepackage{color} or \usepackage{xcolor}
	basicstyle=\ttfamily,        % the size of the fonts that are used for the code
	breakatwhitespace=false,         % sets if automatic breaks should only happen at whitespace
	breaklines=true,                 % sets automatic line breaking
	captionpos=b,                    % sets the caption-position to bottom
	commentstyle=\color{commentsColor}\textit,    % comment style
	extendedchars=true,              % lets you use non-ASCII characters; for 8-bits encodings only, does not work with UTF-8
	frame=tblr,	% adds a frame around the code
	% framexleftmargin=1.5em,
	keepspaces=true,                 % keeps spaces in text, useful for keeping indentation of code (possibly needs columns=flexible)
	keywordstyle=\color{keywordsColor}\bfseries,       % keyword style
	language=Java,                   % the language of the code (can be overrided per snippet)
	otherkeywords={*,...},           % if you want to add more keywords to the set
	numbers=none,                    % where to put the line-numbers; possible values are (none, left, right)
	numbersep=5pt,                   % how far the line-numbers are from the code
	numberstyle=\tiny\color{commentsColor}, % the style that is used for the line-numbers
	rulecolor=\color{black},         % if not set, the frame-color may be changed on line-breaks within not-black text (e.g. comments (green here))
	showspaces=false,                % show spaces everywhere adding particular underscores; it overrides 'showstringspaces'
	showstringspaces=false,          % underline spaces within strings only
	showtabs=false,                  % show tabs within strings adding particular underscores
	stepnumber=1,                    % the step between two line-numbers. If it's 1, each line will be numbered
	stringstyle=\color{stringColor}, % string literal style
	tabsize=2,	                   % sets default tabsize to 2 spaces
	% title=Solution to the Longest increasing subsequence problem,
	% show the filename of files included with \lstinputlisting; also try caption instead of title
	columns=fixed                    % Using fixed column width (for e.g. nice alignment)
}
\begin{document}
\header{Luis Gascon, Ethan Webb, Femi Dosumu}{3}
\textbf{Exercise 1} \vspace{2mm} \\
Analyze the following recurrences using the method that is indicated. In case you use the Master Theorem, state what the corresponding values of \(a, \; b,\) and \(f(n)\) are and how you determined which case of the theorem applies.
\begin{itemize}
	\item  $T(n) = 3 T(\frac{n}{4}) + 3$. Use the Master Theorem to find a $\Theta(\cdot)$ evaluation, or say "Master Theorem cannot be used", if this is the case.
	      \begin{align*}
		      a = 3 \: & b = 4 \: f(n) = 3                                        \\
		               & n^{\log_4{3}} \text{ vs. } 3                             \\
		               & n^{\log_4{3}} > 3                  & \text{case 1 holds} \\
		               & \therefore \Theta{(n^{\log_4{3}})}
	      \end{align*}

	\item  $T(n) = 2 T(\frac{n}{2}) + 3n$. Use the Master Theorem to find a $\Theta(\cdot)$ evaluation, or say "Master Theorem cannot be used", if this is the case.
	      \begin{align*}
		      a = 2 \:      & b = 2 \: f(n) = 3n                        \\
		                    & n^{\log_{2}2} = n                         \\
		                    & n \text{ vs. } 3n   & \text{case 2 holds} \\
		      \therefore \: & \Theta{(n \log{n})}
	      \end{align*}
	\item  $T(n) = 9 T(\frac{n}{3}) + n^2 \log n $. Use the Master Theorem to find a $\Theta(\cdot)$ evaluation, or say "Master Theorem cannot be used", if this is the case.
	      \begin{center}
		      "Master Theorem cannot be used"
	      \end{center}
\end{itemize}
\newpage
\textbf{Exercise 2}
\begin{itemize}
	\item $T(n) = 2T(n-1) + 1$, $T(0)=1$.  Use the iteration method to find a $\Theta(\cdot)$ evaluation for $T(n)$.
	      \begin{align*}
		      T(1) & = 2(1) + 1                   \\
		      T(2) & = 2(2\cdot 1 + 1) + 1        \\
		      T(3) & = 2(2(2\cdot 1 + 1) + 1) + 1 \\
		           & \therefore \Theta{(2^{n})}
	      \end{align*}
	\item $T(n) = T(n-1) + 1$,  $T(0)=1$.  Use the iteration method to find a $\Theta(\cdot)$ evaluation for $T(n)$.
	      \begin{align*}
		      T(1) & = 1 + 1                \\
		      T(2) & = (1+1) + 1            \\
		      T(3) & = ((1+1)+1) + 1        \\
		           & \therefore \Theta{(n)}
	      \end{align*}
	\item Give a $\Theta(\cdot)$ evaluation for the runtime of the following code:

	      \begin{lstlisting}[numbers=none, keywordstyle=\bfseries, frame=none, title=\(\Theta{(n \log{n})}\)]
i = n
while(i >= 1) {
	for (j=1; j <= n; j++)
			x = x+1
	i = i/2
}
	\end{lstlisting}

	\item Give a $\Theta( \cdot)$ evaluation for the runtime of the following code:
	      \begin{lstlisting}[numbers=none, keywordstyle=\bfseries, frame=none, title={\(\Theta{(n \log{n})}\)}]
i = n
while(i >= 1) {
	for (j=1; j <= i; j++)
		x = x+1
	i = i/2
}
	\end{lstlisting}
\end{itemize} \par
\newpage
To obtain the correct amount of star pairs, we only had to modify the merge() function so that it increments the star-pair value if it meets the condition for a pair to be a star pair.

\begin{figure}
	\centering
	\begin{minipage}{\textwidth}
		\raisebox{-0.5\height}{\includegraphics*[width=.5\linewidth]{merge.png}}
		\raisebox{-0.5\height}{\includegraphics*[width=.5\linewidth]{mergeSort.png}}
	\end{minipage}	
\end{figure}
\end{document}
