\documentclass[11pt]{article}
\usepackage{amsfonts}
\usepackage{amsmath}
\usepackage{latexsym}
\usepackage{hyperref}
\usepackage{pdfpages}
\usepackage{tikz}
\usepackage{wrapfig}
\usepackage{fancyvrb}
\usepackage{fontspec}
\usepackage{fancybox}
\usepackage{listings}
\usepackage{enumitem}
\usepackage{ducksay}
\usepackage{xcolor}
\usepackage{amssymb}
\usepackage{graphicx}
\usepackage{parskip}
\usepackage{tabularray}
\usepackage{subcaption}
\usepackage{tikz-qtree}
\usetikzlibrary{automata,arrows}
\graphicspath{ {./images/} }
\setlength{\oddsidemargin}{.25in}
\setlength{\evensidemargin}{.25in}
\setlength{\textwidth}{6in}
\setlength{\topmargin}{-0.4in}
\setlength{\textheight}{8.5in}
\setlength{\parindent}{0cm}
\UseTblrLibrary{booktabs}

\def\squarebox#1{\hbox to #1{\hfill\vbox to #1{\vfill}}}
\def\qed{\hspace*{\fill}
  \vbox{\hrule\hbox{\vrule\squarebox{.667em}\vrule}\hrule}}
\newenvironment{solution}{\begin{trivlist}\item[]{\bf Solution:}}
  {\qed \end{trivlist}}
\newenvironment{solsketch}{\begin{trivlist}\item[]{\bf Solution Sketch:}}
  {\qed \end{trivlist}}
\newenvironment{proof}{\begin{trivlist}\item[]{\bf Proof:}}
  {\qed \end{trivlist}}

\newtheorem{theorem}{Theorem}
\newtheorem{corollary}[theorem]{Corollary}
\newtheorem{lemma}[theorem]{Lemma}
\newtheorem{observation}[theorem]{Observation}
\newtheorem{remark}[theorem]{Remark}
\newtheorem{proposition}[theorem]{Proposition}
\newtheorem{definition}[theorem]{Definition}
\newtheorem{Assertion}[theorem]{Assertion}
\newtheorem{fact}[theorem]{Fact}
\newtheorem{hypothesis}[theorem]{Hypothesis}
% \newtheorem{observation}[theorem]{Observation}
% \newtheorem{proposition}[theorem]{Proposition}
\newtheorem{claim}[theorem]{Claim}
\newtheorem{assumption}[theorem]{Assumption}

% Put more macros here, as needed.
\newcommand{\al}{\alpha}
\newcommand{\Z}{\mathbb Z}
\newcommand{\jac}[2]{\left(\frac{#1}{#2}\right)}
\newcommand{\set}[1]{\#1}
\newcommand{\evenSpace}{\vspace*{\stretch{1}}}
% Assignment header with the appropriate information
% 1st arg: Group member names
% 2nd arg: Assignment #
\newcommand{\header}[2]{
  \begin{center}
    \setlength\fboxsep{.3cm}
    \doublebox{
      \parbox{\dimexpr\linewidth-2\fboxsep-2\fboxrule} {
        #1 \\
        COSC 336 \\
        \today \par
        \centering{\huge{Assignment #2}}
      }}
  \end{center}
}

\def\ppt{{\sf PPT}}
\def\poly{{\sf poly}}
\def\negl{{\sf negl}}
\def\owf{{\sf OWF}}
\def\owp{{\sf OWP}}
\def\tdp{{\sf TDP}}
\def\prg{{\sf PRG}}
\def\prf{{\sf PRF}}
\definecolor{variableColor}{HTML}{AA7700}
\definecolor{commentsColor}{rgb}{0.497495, 0.497587, 0.497464}
\definecolor{keywordsColor}{rgb}{0.00000, 0.000000, 1.500000}
\definecolor{stringColor}{rgb}{0.558215, 0.000000, 0.135316}
\lstset {
  backgroundcolor=\color{white},   % choose the background color; you must add \usepackage{color} or \usepackage{xcolor}
  basicstyle=\ttfamily,        % the size of the fonts that are used for the code
  breakatwhitespace=false,         % sets if automatic breaks should only happen at whitespace
  breaklines=true,                 % sets automatic line breaking
  captionpos=b,                    % sets the caption-position to bottom
  commentstyle=\color{commentsColor}\textit,    % comment style
  extendedchars=true,              % lets you use non-ASCII characters; for 8-bits encodings only, does not work with UTF-8
  frame=tblr, % adds a frame around the code
  % framexleftmargin=1.5em,
  keepspaces=true,                 % keeps spaces in text, useful for keeping indentation of code (possibly needs columns=flexible)
  keywordstyle=\color{keywordsColor}\bfseries,       % keyword style
  language=Java,                   % the language of the code (can be overrided per snippet)
  otherkeywords={*,...},           % if you want to add more keywords to the set
  numbers=none,                    % where to put the line-numbers; possible values are (none, left, right)
  numbersep=5pt,                   % how far the line-numbers are from the code
  numberstyle=\tiny\color{commentsColor}, % the style that is used for the line-numbers
  rulecolor=\color{black},         % if not set, the frame-color may be changed on line-breaks within not-black text (e.g. comments (green here))
  showspaces=false,                % show spaces everywhere adding particular underscores; it overrides 'showstringspaces'
  showstringspaces=false,          % underline spaces within strings only
  showtabs=false,                  % show tabs within strings adding particular underscores
  stepnumber=1,                    % the step between two line-numbers. If it's 1, each line will be numbered
  stringstyle=\color{stringColor}, % string literal style
  tabsize=2,                   % sets default tabsize to 2 spaces
  % title=Solution to the Longest increasing subsequence problem,
  % show the filename of files included with \lstinputlisting; also try caption instead of title
  columns=fixed                    % Using fixed column width (for e.g. nice alignment)
}
\begin{document}
\header{Luis Gascon, Ethan Webb, Femi Dosumu}{8}
\textbf{Exercise 1.}

Describe in plain English (a short paragraph with at most 5-6 lines should be enough) an  algorithm for the following task:
\medskip

\emph{Input}: A directed graph $G = (V,E)$, and a node $u \in V$.

\emph{Goal:}  Output 1 if there is a path from  every  $v \in G$ to $u$ (so if $u$ is reachable from any other node), and output $0$ otherwise.
\medskip

Your algorithm should have runtime $O(n+m)$.  (Hint: Use an idea that we have seen for similar connectivity problems in directed graphs.)
\medskip

Our algorithm starts off by creating an array that stores the node representation, so we'll call it \verb|nodeArr|. We then reverse the direction of the edges in our graph, which should take at most \(O(1)\) time complexity. We can either traverse the graph breadth-first or depth-first, but for our algorithm, we'll implement the traversal using DFS. We'll create another array the size of \verb|nodeArr|, called \verb|rArr| to store the values of the traversed nodes. If the contents of \verb|nodeArr| is equal to that of \verb|rArr| then we return 1, else we return 0. This algorithm should have a time complexity of \(O(n+m)\) because the graph traversal is our slowest operation.
\newpage

\textbf{Exercise 2.}

We have seen that Dijkstra's algorithm can be implemented in two ways: Variant (a) uses an array to store the $dist[]$ values of the unknown nodes, and Variant (b) uses a MIN-HEAP to store these values.
\medskip

(a) Suppose in your application $m \le 3n$. Which variant gives a faster runtime?  Justify your answer.

Variant (b) would give us a faster runtime since our application uses a sparse graph.

\begin{align*}
  m &\leq 3n \\
  m &= O(n)
\end{align*}

A min-heap provides us a run time of \(\Theta{(E \log{V})}\) since we perform at most \(E \times\)min heap operations, where each operation has a run time of \(O(\log{n})\).


(b) Suppose in your application $m \ge n^2/3$. Which variant gives a faster runtime?  Justify your answer.

\medskip

Variant (a) would give us a faster runtime since our application uses a dense graph. 

\begin{align*}
m &\geq \frac{n^2}{3} \\
m &= \Omega{(n^2)} \\
\end{align*}

Using an array to store the \verb|dist| values is actually faster as it results in a run time of \(O(V^{2})\), regardless of how dense the graph is, while a min-heap results in a run time of \(O(V^{2}\log{V})\).

\newpage


Our algorithm starts off by initializing two integer arrays \verb|dist| and \verb|npath|. I've set the undiscovered nodes to have a distance of infinity, and to represent that, I've set the values to the the maximum integer value that Java supports. Using BFS to traverse through our graph helps us identify the shortest path.

\begin{lstlisting}
int[] dist = new int[g.n];
int[] npath = new int[g.n];

for (int i = 0; i < g.n; i++) {
    dist[i] = Integer.MAX_VALUE;
    npath[i] = 0;
}
\end{lstlisting}

A queue is used for our BFS traversal where we dequeue the current node and store that value to \verb|p|. We then traverse the neighboring nodes of \verb|p| to check if that neighboring node has been visited.

If the neighboring node hasn't been visited, we set the distance of the neighboring node to the distance that the predecessor node is currently in, plus 1. The \verb|npath| value is set to the \verb|npath| value that the predecessor node has.
\begin{lstlisting}
while (!queue.isEmpty()) {
  int p = queue.poll();
  for (int v : neighbors) {
      if (dist[v] == Integer.MAX_VALUE) {
          dist[v] = dist[p] + 1;
          npath[v] = npath[p];
          queue.add(v);
      }
          ...
}
\end{lstlisting}

If the neighboring node has been visited, and it's distance is equal to that of the proceeding node, plus 1, then we add the npath value of the proceeding node since it's a path from predecessor to neighbor.
\begin{lstlisting}
  else if (dist[v] == dist[p] + 1)
    npath[v] += npath[p];
\end{lstlisting}

Lastly, we just output the values of arrays \verb|dist| \& \verb|npath|.
\begin{lstlisting}
for (int index = 1; index < g.n; index++)
System.out.printf("dist[%d] = %d \t npath[%d] = %d%n",
        index, dist[index], index, npath[index]);
\end{lstlisting}
\newpage
The algorithm in its entirety.
\begin{lstlisting}
static void getShortestPath(Adj_List_Graph g, int s) {
  int[] dist = new int[g.n];
  int[] npath = new int[g.n];

  for (int i = 0; i < g.n; i++) {
      dist[i] = Integer.MAX_VALUE;
      npath[i] = 0;
  }

  Queue<Integer> queue = new LinkedList<Integer>();

  dist[s] = 0;
  npath[s] = 1;
  queue.add(s);
  while (!queue.isEmpty()) {
      int p = queue.poll();
      List<Integer> neighbors = g.adj.get(p);
      for (int v : neighbors) {
          if (dist[v] == Integer.MAX_VALUE) {
              dist[v] = dist[p] + 1;
              npath[v] = npath[p];
              queue.add(v);
          }

          else if (dist[v] == dist[p] + 1)
              npath[v] += npath[p];

      }
  }

  for (int index = 1; index < g.n; index++)
      System.out.printf("dist[%d] = %d \t npath[%d] = %d%n",
              index, dist[index], index, npath[index]);

}
\end{lstlisting}

Program output can be found on the next page.
\newpage
\begin{lstlisting}
G1 results:

dist[2] = 1 	 npath[2] = 1
dist[3] = 1 	 npath[3] = 1
dist[4] = 1 	 npath[4] = 1
dist[5] = 2 	 npath[5] = 2
dist[6] = 2 	 npath[6] = 1
dist[7] = 3 	 npath[7] = 3

G2 results:

dist[2] = 1 	 npath[2] = 1
dist[3] = 1 	 npath[3] = 1
dist[4] = 1 	 npath[4] = 1
dist[5] = 1 	 npath[5] = 1
dist[6] = 1 	 npath[6] = 1
dist[7] = 2 	 npath[7] = 5
dist[8] = 3 	 npath[8] = 5
dist[9] = 3 	 npath[9] = 5
\end{lstlisting}
\end{document}
