\documentclass[11pt]{article}
\usepackage{amsfonts}
\usepackage{amsmath}
\usepackage{latexsym}
\usepackage{hyperref}
\usepackage{pdfpages}
\usepackage{tikz}
\usepackage{wrapfig}
\usepackage{fancyvrb}
\usepackage{fontspec}
\usepackage{fancybox}
\usepackage{listings}
\usepackage{enumitem}
\usepackage{ducksay}
\usepackage{xcolor}
\usepackage{amssymb}
\usepackage{graphicx}
\usepackage{parskip}
\usepackage{tabularray}
\usepackage{subcaption}
\graphicspath{ {./images/} }
\setlength{\oddsidemargin}{.25in}
\setlength{\evensidemargin}{.25in}
\setlength{\textwidth}{6in}
\setlength{\topmargin}{-0.4in}
\setlength{\textheight}{8.5in}
\setlength{\parindent}{0cm}
\UseTblrLibrary{booktabs}

\def\squarebox#1{\hbox to #1{\hfill\vbox to #1{\vfill}}}
\def\qed{\hspace*{\fill}
	\vbox{\hrule\hbox{\vrule\squarebox{.667em}\vrule}\hrule}}
\newenvironment{solution}{\begin{trivlist}\item[]{\bf Solution:}}
		{\qed \end{trivlist}}
\newenvironment{solsketch}{\begin{trivlist}\item[]{\bf Solution Sketch:}}
		{\qed \end{trivlist}}
\newenvironment{proof}{\begin{trivlist}\item[]{\bf Proof:}}
		{\qed \end{trivlist}}

\newtheorem{theorem}{Theorem}
\newtheorem{corollary}[theorem]{Corollary}
\newtheorem{lemma}[theorem]{Lemma}
\newtheorem{observation}[theorem]{Observation}
\newtheorem{remark}[theorem]{Remark}
\newtheorem{proposition}[theorem]{Proposition}
\newtheorem{definition}[theorem]{Definition}
\newtheorem{Assertion}[theorem]{Assertion}
\newtheorem{fact}[theorem]{Fact}
\newtheorem{hypothesis}[theorem]{Hypothesis}
%\newtheorem{observation}[theorem]{Observation}
%\newtheorem{proposition}[theorem]{Proposition}
\newtheorem{claim}[theorem]{Claim}
\newtheorem{assumption}[theorem]{Assumption}

%Put more macros here, as needed.
\newcommand{\al}{\alpha}
\newcommand{\Z}{\mathbb Z}
\newcommand{\jac}[2]{\left(\frac{#1}{#2}\right)}
\newcommand{\set}[1]{\{#1\}}
\newcommand{\evenSpace}{\vspace*{\stretch{1}}}
% Assignment header with the appropriate information
% 1st arg: Group member names
% 2nd arg: Assignment #
\newcommand{\header}[2]{
	\begin{center}
		\setlength\fboxsep{.3cm}
		\doublebox{
			\parbox{\dimexpr\linewidth-2\fboxsep-2\fboxrule} {
				#1 \\
				COSC 336 \\
				\today \par
				\centering{\huge{Assignment #2}}
			}}
	\end{center}
}

\def\ppt{{\sf PPT}}
\def\poly{{\sf poly}}
\def\negl{{\sf negl}}
\def\owf{{\sf OWF}}
\def\owp{{\sf OWP}}
\def\tdp{{\sf TDP}}
\def\prg{{\sf PRG}}
\def\prf{{\sf PRF}}
\definecolor{variableColor}{HTML}{AA7700}
\definecolor{commentsColor}{rgb}{0.497495, 0.497587, 0.497464}
\definecolor{keywordsColor}{rgb}{0.00000, 0.000000, 1.500000}
\definecolor{stringColor}{rgb}{0.558215, 0.000000, 0.135316}
\lstset{
	backgroundcolor=\color{white},   % choose the background color; you must add \usepackage{color} or \usepackage{xcolor}
	basicstyle=\ttfamily,        % the size of the fonts that are used for the code
	breakatwhitespace=false,         % sets if automatic breaks should only happen at whitespace
	breaklines=true,                 % sets automatic line breaking
	captionpos=b,                    % sets the caption-position to bottom
	commentstyle=\color{commentsColor}\textit,    % comment style
	extendedchars=true,              % lets you use non-ASCII characters; for 8-bits encodings only, does not work with UTF-8
	frame=tblr, % adds a frame around the code
	% framexleftmargin=1.5em,
	keepspaces=true,                 % keeps spaces in text, useful for keeping indentation of code (possibly needs columns=flexible)
	keywordstyle=\color{keywordsColor}\bfseries,       % keyword style
	language=Java,                   % the language of the code (can be overrided per snippet)
	otherkeywords={*,...},           % if you want to add more keywords to the set
	numbers=none,                    % where to put the line-numbers; possible values are (none, left, right)
	numbersep=5pt,                   % how far the line-numbers are from the code
	numberstyle=\tiny\color{commentsColor}, % the style that is used for the line-numbers
	rulecolor=\color{black},         % if not set, the frame-color may be changed on line-breaks within not-black text (e.g. comments (green here))
	showspaces=false,                % show spaces everywhere adding particular underscores; it overrides 'showstringspaces'
	showstringspaces=false,          % underline spaces within strings only
	showtabs=false,                  % show tabs within strings adding particular underscores
	stepnumber=1,                    % the step between two line-numbers. If it's 1, each line will be numbered
	stringstyle=\color{stringColor}, % string literal style
	tabsize=2,                   % sets default tabsize to 2 spaces
	% title=Solution to the Longest increasing subsequence problem,
	% show the filename of files included with \lstinputlisting; also try caption instead of title
	columns=fixed                    % Using fixed column width (for e.g. nice alignment)
}
\begin{document}
\header{Luis Gascon, Ethan Webb, Femi Dosumu}{5}
Similar to the longest increasing subsequence problem, we use dynamic programming to solve the problem more efficiently by minimizing the number of calculations we have to perform. We approach this problem bottom up as it this problem has a case of overlapping subproblems. We store the solutions to each subproblem to an array called \lstinline{s[]}. We also have to find a way of returning a subsequence that sums up to the maximum sum, so to achieve that solution, we'll have to keep track of the elements, specifically, their indices as the we iterate through the sequence.

We iterate the sequence with an inner loop with the $j$ index as inner and the $i$ index as outer where \(i > j\) 

To populate \lstinline{s[]}:
\begin{lstlisting}[frame=none]
  if arr[i] >= arr[j]:
    s[i] = max(s[i], s[j]+arr[i])
\end{lstlisting}
In order to get the elements of the sequence that sum up to the maximum sum of increasing subsequence, we have to create another sequence called \lstinline{p[]} that keeps track of indices where the addition happens.

To populate \lstinline{p[]}:
\begin{lstlisting}[frame=none]
  if s[i] < s[j] + s[i]:
    p[i] = j
\end{lstlisting}
\newpage
To have our algorithm return the maximum sum and its respective subsequence, we created a class called \lstinline{Outputs}, which has the attributes \lstinline{sum} and \lstinline{sub_sequence}.
To construct the subsequence, we create a function called \lstinline{consruct_subsequence()} which has the parameters \lstinline{sequence, N - 1} and \lstinline{p} that returns the subsequence that sum up to the maximum sum.
\begin{lstlisting}
  static List<Integer> construct_subsequence(List<Integer> arr, int N, List<Integer> p) {
    // Return value
    List<Integer> subsequence = new ArrayList<>();
    // indices[] stores the target indices of the subsequence
    List<Integer> indices = new ArrayList<>();
    indices.add(N);
    get_predecessor_indices(N, p, indices);
    indices.forEach(x -> subsequence.add(arr.get(x)));
    // Reverse the list in place for an increasing subsequence
    Collections.reverse(subsequence);
    return subsequence;
  }
\end{lstlisting}
To backtrack from the index of the maximum sum to the index of the subproblem solutions, we create a recursive function called \lstinline{get_predecessor_indices(index, p, and sum_index)} where the base case is when the return value is the same as the value in \lstinline{p}.
\begin{lstlisting}
  static int get_predecessor_indices(int index, List<Integer> p, List<Integer> sum_index) {

    // Base case
    if (p.get(index) == index)
      return index;

    sum_index.add(p.get(index));
    return get_predecessor_indices(p.get(index), p, sum_index);
  }
\end{lstlisting}
\newpage
The algorithm's complete code:
\begin{lstlisting}
static Outputs max_sum(int N, List<Integer> sequence) {
    // s[] stores the sums at each index i. Initialized with values of sequence
    // p[] stores the indices of the elements that sum up to the maximum sum. Initialized with values from range of 0 to N - 1
    List<Integer> s = new ArrayList<>(sequence);
    List<Integer> p = new ArrayList<>();

    for (int index = 0; index < N; index++)
      p.add(index);

    for (int i = 1; i < N; i++)
      for (int j = 0; j < i; j++) {
        if (sequence.get(i) >= sequence.get(j)) {
          // Populate p[]
          if (s.get(i) < s.get(j) + sequence.get(i)) {
            p.set(i, j);
          }
          // Populate s[]
          s.set(i, Math.max(s.get(i), s.get(j) + sequence.get(i)));
        }
      }

    return new Outputs(s.stream().max((a, b) -> a - b).get(), construct_subsequence(sequence, N - 1, p));
  }
\end{lstlisting}

\begin{center}
  \large{Solutions obtained} \vspace{5mm} \\
	\begin{tblr}{hlines, vlines}
		Input     & Sum     & Subsequence                           \\
		input-5.3 & 130,021 & [ 4601, 20255, 23073, 32092, 50000 ]  \\
		input-5.4 & 143,418 & [ 25197, 26355, 29960, 30953, 30953 ] \\
	\end{tblr}
\end{center}
\end{document}
