\documentclass[11pt]{article}
\usepackage{amsfonts}
\usepackage{amsmath}
\usepackage{latexsym}
\usepackage{hyperref}
\usepackage{pdfpages}
\usepackage{tikz}
\usepackage{fontspec}
\usepackage{fancybox}
\usepackage{tcolorbox}
\usepackage{listings}
\usepackage{enumitem}
\usepackage{ducksay}
\usepackage{xcolor}
\usepackage{tabularx}
\usepackage{amssymb}
\usepackage{tabularray}
\usepackage[parfill]{parskip}
\usepackage[makeroom]{cancel}

\setlength{\oddsidemargin}{.25in}
\setlength{\evensidemargin}{.25in}
\setlength{\textwidth}{6in}
\setlength{\topmargin}{-0.4in}
\setlength{\textheight}{8.5in}

\def\squarebox#1{\hbox to #1{\hfill\vbox to #1{\vfill}}}
\def\qed{\hspace*{\fill}
        \vbox{\hrule\hbox{\vrule\squarebox{.667em}\vrule}\hrule}}
\newenvironment{solution}{\begin{trivlist}\item[]{\bf Solution:}}
                      {\qed \end{trivlist}}
\newenvironment{solsketch}{\begin{trivlist}\item[]{\bf Solution Sketch:}}
                      {\qed \end{trivlist}}
\newenvironment{proof}{\begin{trivlist}\item[]{\bf Proof:}}
                      {\qed \end{trivlist}}

\newtheorem{theorem}{Theorem}
\newtheorem{corollary}[theorem]{Corollary}
\newtheorem{lemma}[theorem]{Lemma}
\newtheorem{observation}[theorem]{Observation}
\newtheorem{remark}[theorem]{Remark}
\newtheorem{proposition}[theorem]{Proposition}
\newtheorem{definition}[theorem]{Definition}
\newtheorem{Assertion}[theorem]{Assertion}
\newtheorem{fact}[theorem]{Fact}
\newtheorem{hypothesis}[theorem]{Hypothesis}
%\newtheorem{observation}[theorem]{Observation
%\newtheorem{proposition}[theorem]{Proposition}
\newtheorem{claim}[theorem]{Claim}
\newtheorem{assumption}[theorem]{Assumption}

%Put more macros here, as needed.
\newcommand{\al}{\alpha}
\newcommand{\Z}{\mathbb Z}
\newcommand{\jac}[2]{\left(\frac{#1}{#2}\right)}
\newcommand{\set}[1]{\{#1\}}
\newcommand{\evenSpace}{\vspace*{\stretch{1}}}
% Assignment header with the appropriate information
% 1st arg: Group member names
% 2nd arg: Assignment #
\newcommand{\header}[2]{
  \begin{center}
	\doublebox{
		\parbox{\textwidth} {
			#1 \\
			COSC 336 \\
			\today \par
			\centering{\huge{Assignment #2}}
		}}
\end{center}
}

\newcommand*\Eval[3]{\left.#1\right\rvert_{#2}^{#3}}

\def\ppt{{\sf PPT}}
\def\poly{{\sf poly}}
\def\negl{{\sf negl}}
\def\owf{{\sf OWF}}
\def\owp{{\sf OWP}}
\def\tdp{{\sf TDP}}
\def\prg{{\sf PRG}}
\def\prf{{\sf PRF}}
\addtolength{\jot}{3mm}
\begin{document}
\header{Luis Gascon, Ethan Webb, Femi Dosumu}{2}
\textbf{Exercise 1}
\begin{enumerate}[label=(\alph*)]
  \item Find a $\Theta$ evaluation for the function \((4n+1)8^{\log{(n^2)}}\).
        \vspace{3mm} \\
        \(\Theta{(n^{7})}\)
        \vspace{3mm}
  \item Give an example of two functions \(t_1(n)\) and \(t_2(n)\) that satisfy the relations: \\ \(t_1(n) = \Theta{(n^2)}, \: t_2(n) = \Theta{(n^2)}\) and \(t_1(n) - t_2(n) = o(n^2)\).
        \begin{flalign*}
           & t_{1}(n) = 3n^{2} + 3n   & \\
           & t_{2}(n) = 3n^{2} - 2    & \\
           & 3n^{2} - 3n^{2} + 3n - 2 & \\
           & 3n-2
        \end{flalign*}
  \item Give an example of a function \(t_1(n)\) such that \(t_1(n) = \Theta{(t_1(2n))}\).
        \vspace{3mm} \\
        \(t_{1}(n) = \displaystyle n^{2}+n-10\) \vspace{2mm}\\
        \(t_{1}(2n) = \displaystyle 2n^{2}+2n-10\) \vspace{2mm} \\
        \(t_{1}(2n) = \Theta{(t_1(2n))}\)
        \vspace{3mm}
  \item Give an example of a function \(t_2(n)\) such that \(t_2(n) = o(t_2(2n))\).
        \vspace{3mm} \\
        \(t_{2}(n) = n^{n} \) \vspace{2mm}\\
        \(t_{2}(2n) = 2n^{2n}\) \vspace{2mm}\\
        \(2n^{2n} > n^n\)
        \vspace{3mm}
\end{enumerate}
\newpage
\textbf{Exercise 2} \vspace{2mm} \\
Indicate whether $A$ is $O$, $o$, $\Omega$, $\omega$, or $\Theta$ of $B$. Answer should be ``yes'' or ``no'' in each box. \\
Assume that \(k \geq 1,\,\varepsilon > 1,\, \) and \(c > 1\) are constants.
\begin{center}
  \begin{tblr}{|X[2,c] X[2,c] |[2pt] X[1,c] | X[1,c] | X[1,c] | X[1,c] | X[1,c] | }
    \hline
    $A$             & $B$               & $O$ & $o$ & $\Omega$ & $\omega$ & \Theta \\ \hline
    {$\log^{k}(n)$} & $n^{\varepsilon}$ & yes & yes & no       & no       & no     \\ \hline
    $n^{k}$         & $c^{n}$           & no  & no  & yes      & yes      & no     \\ \hline
    $2^{n}$         & $2^{n/2}$         & no  & no  & yes      & yes      & no     \\ \hline
    $n^{\log{c}}$   & $c^{\log{n}}$     & yes & no  & yes      & no       & yes    \\ \hline
    $\log{n!}$      & $\log{n^n}$       & yes & no  & yes      & no       & yes    \\ \hline
  \end{tblr}
\end{center}
\newpage
{\parindent0pt\textbf{Exercise 3}} \vspace{2mm} \\
For each of the following program fragments, give a $\Theta(\cdot)$ estimation of the running time as a function of $n$.
\begin{itemize}
  \item[(a)]
    \begin{verbatim}
sum = 0;
for (int i = 0; i< n * n; i++) {
    for(int j =0;  j < n/2; j++)
        sum++;
}
\end{verbatim}
  \item[(b)]
    \begin{verbatim}
sum = 0;
for (int i = 0; i< n; i++) {
    sum++;}

for(int j = 0;  j < n/2; j++){
    sum++;}
\end{verbatim}
  \item[(c)]
    \begin{verbatim}
sum = 0;
for (int i = 0; i< n * n; i++) {
    for(int j = 0;  j < n * n; j++)
        sum++
}
\end{verbatim}
  \item[(d)]
    \begin{verbatim}
sum = 0;
for (int i = 1; i< n; i = 2*i)
        sum++
\end{verbatim}

  \item[(e)]
    \begin{verbatim}
sum = 0;
for (int i = 0; i< n; i++) {
    for(int j = 1;  j < n * n; j = 2*j)
        sum++
}
\end{verbatim}
\end{itemize}
\vspace{2cm}
\begin{enumerate}[label=(\alph*), leftmargin=*]
  \item \(\Theta{(n^2)}\)
  \item \(\Theta{(n)}\)
  \item \(\Theta{(n^2)}\)
  \item \(\Theta{(\log n)}\)
  \item \(\Theta{(n\log n)}\)
\end{enumerate}
\newpage
{\parindent0pt\textbf{Exercise 4}}
\begin{enumerate}[label=(\alph*)]
  \item Compute the sum \(S_1 = 500 + 501 + 502 + 503 + \ldots + 999\) (the sum of all integers from $500$ to $999$).
        \begin{align*}
          N   & = 500                                 \\
          S_1 & = \displaystyle\frac{N(500 + 999)}{2} \\
              & = 374,750
        \end{align*}
  \item Compute the sum \(S_2 = 1 + 3 + 5 + \ldots + 999\) (the sum of all odd integers from $1$ to $999$).
        \begin{align*}
          S_2 & = \displaystyle\frac{500(1 + 999)}{2} \\
              & = 250,000
        \end{align*}
  \item A group of $30$ persons need to form a committee of $3$ person. How many such committees are possible?
        \begin{align*}
          \binom{30}{3} & = \displaystyle\frac{30!}{(30-3)!\cdot 3!} \\
                        & = 4,060
        \end{align*}
        4,060 committees are possible.
  \item Let $C_n$ be the number of committees of $4$ persons selected from a group of $n$ persons. Is the estimation \(C_n=o(n^3)\) correct? Justify your answer.
        \begin{align*}
          \binom{n}{4} & = \displaystyle \frac{n!}{(n-4)! \cdot 4!}                                                             \\
                       & = \displaystyle \frac{n(n-1)(n-2)(n-3)\cancel{(n-4)(n-5)} \ldots}{\cancel{(n-4)(n-5) \ldots} \cdot 4!} \\
                       & = \displaystyle \frac{n(n-1)(n-2)(n-3)}{4!}                                                            \\
                       & = \Theta{(n^{4})}
        \end{align*}
        The estimation of \(C_{n} = o(n^{3})\) is incorrect since \(n^4 > n^3\).
\end{enumerate}
\newpage
{\parindent0pt\textbf{Exercise 5}} \vspace{2mm} \\
Find a $\Theta(\cdot)$ evaluation for the sum
\[
  S = 1\sqrt{1} + 2\sqrt{2} + \ldots + n\sqrt{n}.
\]
Find a function $f$ such that \(S = \Theta{(f(n))}\)
\vspace{3mm} \\
The summation is monotonically increasing, so:
\[
  \displaystyle \int_{0}^{n} f(x) \, dx \leq S_{n} \leq \displaystyle \int_{1}^{n+1} f(x) \, dx
\]
\begin{flalign*}
  &\text{Lower bound}  && \text{Upper bound}\\
  \displaystyle \int_{0}^{n}x \; \sqrt[]{x}  \, dx & = \displaystyle \int_{0}^{n} x^{3/2} \, dx & \displaystyle \int_{1}^{n+1} x \; \sqrt[]{ x } \, dx &= \Eval{\dfrac{2}{5}x^{5/2}}{1}{n+1} \\
                                                   & = \displaystyle \dfrac{x^{(3/2) + 1}}{3/2 + 1}  &&= \dfrac{2}{5}(n+1)^{5/2} - \dfrac{2}{5}(1)^{5/2}  \\
                                                   &= \displaystyle \Eval{\dfrac{2}{5}x^{5/2}}{0}{n} &&= O(n^{5/2})  \\  
                                                   &= \dfrac{2}{5}n^{5/2} - 0 \\
                                                   &= \Theta{(n^{5/2})}
\end{flalign*}
\[\therefore \Theta{(n^{5/2})}\]
\end{document}