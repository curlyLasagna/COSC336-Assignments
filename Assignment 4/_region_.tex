\message{ !name(solution.tex)}\documentclass[11pt]{article}
\usepackage{amsfonts}
\usepackage{amsmath}
\usepackage{latexsym}
\usepackage{hyperref}
\usepackage{pdfpages}
\usepackage{tikz}
\usepackage{fontspec}
\usepackage{fancybox}
\usepackage{listings}
\usepackage{enumitem}
\usepackage{ducksay}
\usepackage{xcolor}
\usepackage{amssymb}

\setlength{\oddsidemargin}{.25in}
\setlength{\evensidemargin}{.25in}
\setlength{\textwidth}{6in}
\setlength{\topmargin}{-0.4in}
\setlength{\textheight}{8.5in}

\def\squarebox#1{\hbox to #1{\hfill\vbox to #1{\vfill}}}
\def\qed{\hspace*{\fill}
        \vbox{\hrule\hbox{\vrule\squarebox{.667em}\vrule}\hrule}}
\newenvironment{solution}{\begin{trivlist}\item[]{\bf Solution:}}
                      {\qed \end{trivlist}}
\newenvironment{solsketch}{\begin{trivlist}\item[]{\bf Solution Sketch:}}
                      {\qed \end{trivlist}}
\newenvironment{proof}{\begin{trivlist}\item[]{\bf Proof:}}
                      {\qed \end{trivlist}}

\newtheorem{theorem}{Theorem}
\newtheorem{corollary}[theorem]{Corollary}
\newtheorem{lemma}[theorem]{Lemma}
\newtheorem{observation}[theorem]{Observation}
\newtheorem{remark}[theorem]{Remark}
\newtheorem{proposition}[theorem]{Proposition}
\newtheorem{definition}[theorem]{Definition}
\newtheorem{Assertion}[theorem]{Assertion}
\newtheorem{fact}[theorem]{Fact}
\newtheorem{hypothesis}[theorem]{Hypothesis}
%\newtheorem{observation}[theorem]{Observation}
%\newtheorem{proposition}[theorem]{Proposition}
\newtheorem{claim}[theorem]{Claim}
\newtheorem{assumption}[theorem]{Assumption}

%Put more macros here, as needed.
\newcommand{\al}{\alpha}
\newcommand{\Z}{\mathbb Z}
\newcommand{\jac}[2]{\left(\frac{#1}{#2}\right)}
\newcommand{\set}[1]{\{#1\}}
% Assignment header with the appropriate information
% 1st arg: Group member names
% 2nd arg: Assignment #
\newcommand{\header}[2]{
  \begin{center}
	\setlength\fboxsep{.3cm}
	\doublebox{
		\parbox{\dimexpr\linewidth-2\fboxsep-2\fboxrule} {
			#1 \\
			COSC 336 \\
			\today \par
			\centering{\huge{Assignment #2}}
		}}
\end{center}
}

\def\ppt{{\sf PPT}}
\def\poly{{\sf poly}}
\def\negl{{\sf negl}}
\def\owf{{\sf OWF}}
\def\owp{{\sf OWP}}
\def\tdp{{\sf TDP}}
\def\prg{{\sf PRG}}
\def\prf{{\sf PRF}}

\definecolor{variableColor}{HTML}{AA7700}
\definecolor{commentsColor}{rgb}{0.497495, 0.497587, 0.497464}
\definecolor{keywordsColor}{rgb}{0.00000, 0.000000, 1.500000}
\definecolor{stringColor}{rgb}{0.558215, 0.000000, 0.135316}
\lstset {
	backgroundcolor=\color{white},   % choose the background color; you must add \usepackage{color} or \usepackage{xcolor}
	basicstyle=\ttfamily,        % the size of the fonts that are used for the code
	breakatwhitespace=false,         % sets if automatic breaks should only happen at whitespace
	breaklines=true,                 % sets automatic line breaking
	captionpos=b,                    % sets the caption-position to bottom
	commentstyle=\color{commentsColor}\textit,    % comment style
	extendedchars=true,              % lets you use non-ASCII characters; for 8-bits encodings only, does not work with UTF-8
	frame=tblr,	% adds a frame around the code
	framexleftmargin=1.5em,
	keepspaces=true,                 % keeps spaces in text, useful for keeping indentation of code (possibly needs columns=flexible)
	keywordstyle=\color{keywordsColor}\bfseries,       % keyword style
	language=Java,                   % the language of the code (can be overrided per snippet)
	otherkeywords={*,...},           % if you want to add more keywords to the set
	numbers=left,                    % where to put the line-numbers; possible values are (none, left, right)
	numbersep=5pt,                   % how far the line-numbers are from the code
	numberstyle=\tiny\color{commentsColor}, % the style that is used for the line-numbers
	rulecolor=\color{black},         % if not set, the frame-color may be changed on line-breaks within not-black text (e.g. comments (green here))
	showspaces=false,                % show spaces everywhere adding particular underscores; it overrides 'showstringspaces'
	showstringspaces=false,          % underline spaces within strings only
	showtabs=false,                  % show tabs within strings adding particular underscores
	stepnumber=1,                    % the step between two line-numbers. If it's 1, each line will be numbered
	stringstyle=\color{stringColor}, % string literal style
	tabsize=2,	                   % sets default tabsize to 2 spaces
	% title=Solution to the Longest increasing subsequence problem,
	% show the filename of files included with \lstinputlisting; also try caption instead of title
	columns=fixed                    % Using fixed column width (for e.g. nice alignment)
}
\begin{document}

\message{ !name(solution.tex) !offset(-3) }

\header{Luis Gascon, Ethan Webb, Femi Dosumu}{4}
\textbf{Exercise 1} \vspace{2mm} \\
For each of the following functions, give a \(\Theta{(t(n))}\) estimation with the simplest possible \(t(n)\).
\begin{enumerate}
	\item \(13n^{2}-2n+56\) = \(\Theta{(n^2)}\)
	\item \(2.5\log{(n)}+2\) = \(\Theta{(\log{n})}\)
	\item \(n(12+\log{n})\) = \(\Theta{(n\log{n})}\)
	\item \(1+2+3+\ldots+2n\) = \(\Theta{(n^2)}\)
	\item \(1+2+3+\ldots+n^2\) = \(\Theta{(n^3)}\)
	\item \(\log{(n^{3})}+10\) = \(\Theta{(\log{n})}\)
	\item \(\log{(n^3)}+n\log{n}\) = \(\Theta{(n\log{n}})\)
	\item \(n\log{(n^3)}+n\log{n}\) = \(\Theta{(n\log{n}})\)
	\item \(2^{2\log{n}}+5n+1\) = \(\Theta{(n^2)}\)
\end{enumerate}
\textbf{Exercise 2} \vspace{2mm}
\begin{enumerate}
	\item Evaluate the following postfix arithmetic expression: \(10 \: 3 \: 4 - 5 * /\)
	      \[-2\]
	\item Convert the following infix arithmetic expression to postfix notation: \((((2+3)*5)-15)\)
	      \[2 \quad 3 \quad + \quad 5 \quad * \quad 15 \quad -\]
\end{enumerate}
\newpage
\noindent
\textbf{Exercise 3} \vspace*{2mm} \\
Consider the following algorithms $A$ and $B$ for the problem of computing \(2^n\pmod{317}\)
\begin{verbatim}
  Algorithm A.

  mod_exp_A(n) {
     if (n == 0) return 1;
     else {
            t = mod_exp_A(n/2);
            if (n is even) return t*t (mod 317);
            if (n is odd) return t*t*2 (mod 317);
  }}
  \end{verbatim}

\begin{verbatim}
  Algorithm B.

  mod_exp_B(n) {
     if (n == 0) return 1;
     else {
            if (n is even)
              return mod_exp_B(n/2) * mod_exp_B(n/2) (mod 317);
            if (n is odd)
              return mod_exp_B(n/2) * mod_exp_B(n/2) *2 (mod 317);
  }}
  \end{verbatim}
\begin{enumerate}
	\item Write the reucrrence for the runtime \(T_{A}(n)\) of algorithm $A$ and solve the recurrence to find a \(\Theta(\cdot)\) estimation of \(T_{A}(n)\)
	      \[T(n) = T\left(\dfrac{n}{2}\right) + 2\]
	\item Write the recurrence for the runtime \(T_B(n)\) of algorithm $B$, and solve the recurrence to find a \(\Theta(\cdot)\) estimation of \(T_B(n)\).
	      \[T(n) = 2T\left(\dfrac{n}{2}\right) + 2\]
	\item Which algorithm is faster? \\
	      \begin{minipage}{.5\linewidth}
		      \begin{gather*}
			      T_{A}(n)            =T\left(\dfrac{n}{2}\right) + 2 \\
			      a = 1 \quad         b = 2 \quad f(n) = 2            \\
			      n^{\log_{2}{1}} \quad  \text{ vs. } \quad 2               \\
			      T_{A}(n) = \Theta{(\log{n})}
		      \end{gather*}
	      \end{minipage}
	      \begin{minipage}{.5\linewidth}
		      \begin{gather*}
			      T_{B}(n)  =T\left(\dfrac{n}{2} + 2\right) + 2 \\
            a = 2 \quad          b = 2 \quad f(n) = 2            \\
            n^{\log_{2}{2}} \quad \text{ vs. } \quad 2 \\
             T_{B}(n)=\Theta{(n)}
           \end{gather*}
         \end{minipage}
        \begin{center}
          \(\therefore T_{A}(n)\) is the faster algorithm.
        \end{center}
\end{enumerate}
\newpage
\noindent
\textbf{Exercise 4} \vspace{2mm} \\
Give a \(\Theta(\cdot)\) evaluation for the runtime of the following code:
\begin{center}
	\begin{lstlisting}[numbers=none, keywordstyle=\bfseries, frame=none, title={Assume that $n$ is a power two. Then $i$ from the outer loop takes successively the values: \(1, \quad 2, \quad 2^2, \quad 2^3, \quad \ldots, 2^{\log{n}}\)}]
      i= 1; x=0;
      while(i <= n) {
            j=1;
            while (j <= i) { x=x+1; j= 2*j; }
            i= 2*i;
     }
    \end{lstlisting}
	\[ \Theta{(\log^{2}{n})} \]
\end{center}
\end{document}

\message{ !name(solution.tex) !offset(-205) }
