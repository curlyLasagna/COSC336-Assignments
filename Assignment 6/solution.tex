\documentclass[11pt]{article}
\usepackage{amsfonts}
\usepackage{amsmath}
\usepackage{latexsym}
\usepackage{hyperref}
\usepackage{pdfpages}
\usepackage{tikz}
\usepackage{wrapfig}
\usepackage{fancyvrb}
\usepackage{fontspec}
\usepackage{fancybox}
\usepackage{listings}
\usepackage{enumitem}
\usepackage{ducksay}
\usepackage{xcolor}
\usepackage{amssymb}
\usepackage{graphicx}
\usepackage{parskip}
\usepackage{tabularray}
\usepackage{subcaption}
\graphicspath{ {./images/} }
\setlength{\oddsidemargin}{.25in}
\setlength{\evensidemargin}{.25in}
\setlength{\textwidth}{6in}
\setlength{\topmargin}{-0.4in}
\setlength{\textheight}{8.5in}
\setlength{\parindent}{0cm}
\UseTblrLibrary{booktabs}

\def\squarebox#1{\hbox to #1{\hfill\vbox to #1{\vfill}}}
\def\qed{\hspace*{\fill}
        \vbox{\hrule\hbox{\vrule\squarebox{.667em}\vrule}\hrule}}
\newenvironment{solution}{\begin{trivlist}\item[]{\bf Solution:}}
                      {\qed \end{trivlist}}
\newenvironment{solsketch}{\begin{trivlist}\item[]{\bf Solution Sketch:}}
                      {\qed \end{trivlist}}
\newenvironment{proof}{\begin{trivlist}\item[]{\bf Proof:}}
                      {\qed \end{trivlist}}

\newtheorem{theorem}{Theorem}
\newtheorem{corollary}[theorem]{Corollary}
\newtheorem{lemma}[theorem]{Lemma}
\newtheorem{observation}[theorem]{Observation}
\newtheorem{remark}[theorem]{Remark}
\newtheorem{proposition}[theorem]{Proposition}
\newtheorem{definition}[theorem]{Definition}
\newtheorem{Assertion}[theorem]{Assertion}
\newtheorem{fact}[theorem]{Fact}
\newtheorem{hypothesis}[theorem]{Hypothesis}
%\newtheorem{observation}[theorem]{Observation}
%\newtheorem{proposition}[theorem]{Proposition}
\newtheorem{claim}[theorem]{Claim}
\newtheorem{assumption}[theorem]{Assumption}

%Put more macros here, as needed.
\newcommand{\al}{\alpha}
\newcommand{\Z}{\mathbb Z}
\newcommand{\jac}[2]{\left(\frac{#1}{#2}\right)}
\newcommand{\set}[1]{\{#1}
\newcommand{\evenSpace}{\vspace*{\stretch{1}}}
% Assignment header with the appropriate information
% 1st arg: Group member names
% 2nd arg: Assignment #
\newcommand{\header}[2]{
\begin{center}
 \setlength\fboxsep{.3cm}
 \doublebox{
  \parbox{\dimexpr\linewidth-2\fboxsep-2\fboxrule} {
    #1 \\
    COSC 336 \\
    \today \par
    \centering{\huge{Assignment #2}}
    }}
\end{center}
}

\def\ppt{{\sf PPT}}
\def\poly{{\sf poly}}
\def\negl{{\sf negl}}
\def\owf{{\sf OWF}}
\def\owp{{\sf OWP}}
\def\tdp{{\sf TDP}}
\def\prg{{\sf PRG}}
\def\prf{{\sf PRF}}
\definecolor{variableColor}{HTML}{AA7700}
\definecolor{commentsColor}{rgb}{0.497495, 0.497587, 0.497464}
\definecolor{keywordsColor}{rgb}{0.00000, 0.000000, 1.500000}
\definecolor{stringColor}{rgb}{0.558215, 0.000000, 0.135316}
\lstset {
  backgroundcolor=\color{white},   % choose the background color; you must add \usepackage{color} or \usepackage{xcolor}
basicstyle=\ttfamily,        % the size of the fonts that are used for the code
breakatwhitespace=false,         % sets if automatic breaks should only happen at whitespace
breaklines=true,                 % sets automatic line breaking
captionpos=b,                    % sets the caption-position to bottom
commentstyle=\color{commentsColor}\textit,    % comment style
extendedchars=true,              % lets you use non-ASCII characters; for 8-bits encodings only, does not work with UTF-8
frame=tblr, % adds a frame around the code
% framexleftmargin=1.5em,
keepspaces=true,                 % keeps spaces in text, useful for keeping indentation of code (possibly needs columns=flexible)
keywordstyle=\color{keywordsColor}\bfseries,       % keyword style
language=Java,                   % the language of the code (can be overrided per snippet)
otherkeywords={*,...},           % if you want to add more keywords to the set
numbers=none,                    % where to put the line-numbers; possible values are (none, left, right)
numbersep=5pt,                   % how far the line-numbers are from the code
numberstyle=\tiny\color{commentsColor}, % the style that is used for the line-numbers
rulecolor=\color{black},         % if not set, the frame-color may be changed on line-breaks within not-black text (e.g. comments (green here))
showspaces=false,                % show spaces everywhere adding particular underscores; it overrides 'showstringspaces'
showstringspaces=false,          % underline spaces within strings only
showtabs=false,                  % show tabs within strings adding particular underscores
stepnumber=1,                    % the step between two line-numbers. If it's 1, each line will be numbered
stringstyle=\color{stringColor}, % string literal style
tabsize=2,                   % sets default tabsize to 2 spaces
% title=Solution to the Longest increasing subsequence problem,
% show the filename of files included with \lstinputlisting; also try caption instead of title
columns=fixed                    % Using fixed column width (for e.g. nice alignment)
}
\begin{document}
\header{Luis Gascon, Ethan Webb, Femi Dosumu}{6}
\textbf{Exercise 1.}  Recall the \textsf{Partition} subroutine employed by \textsf{QuickSort}. You are told that the following array has been partitioned around some pivot element:
\medskip

\begin{center}
  \begin{tabular}{|c|c|c|c|c|c|c|c|c|}
    \hline
    3 & 1 & 2 & 4 & 5 & 8 & 7 & 6 & 9 \\
    \hline
  \end{tabular}
  \medskip
\end{center}


Which of the elements could have been the pivot element? (List all that apply; there could be more than one possibility.)

\textbf{Answer:} 4, 5, and 9 are all possible pivots. \vspace{2mm}\\
\textbf{Exercise 2.}
Let $\alpha$ be some constant, independent of the input array length $n$, strictly between $0$ and $1/2$. What is the probability that, with a randomly chosen pivot element, the \textsf{Partition} function produces a split in which the size of both the resulting subproblems is at least $\alpha \cdot n$. Choose the answer from the following list and justify your answer.

\begin{enumerate}[label=(\alph*)]
  \item $\alpha$
  \item $1 - \alpha$
  \item $1-2\alpha$
  \item $2 - 2 \alpha$
\end{enumerate}

\textbf{Answer:} (c) 1-2a
\newline
The pivot must be greater or equal to a\cdot n (left side of the pivot), meaning the right side of the pivot must be less than or equal to n - a\cdot n. We then just compute this probability and do some simple cancellations.

\[
  a\cdot n \le pivot \le n-a\cdot n
\]
\[
  P(a\cdot n \le pivot \le n-a\cdot n)
\]
\[
  P(A) = \frac{n-2a\cdot n}{n}
\]
\[
  P(A) = 1-2a
\]
\newpage

We augment the node class by adding an integer property of size, which counts the number of descendant a node has, plus itself. To increment size, we modify the recursive insert helper function to increment \lstinline|root.size| by 1.
We utilize the size property by using it as a number to be compared with $k$. We first initialize a variable \lstinline{leftSize}, which stores the size of the current node's left child. \\
We have 3 cases that determines whether our algorithm recurses at a certain direction or return and we have found the solution.

\begin{itemize}
  \item Case 1: $k = leftSize + 1$
        \begin{itemize}
          \item[--] This is the case when we've found the k-th smallest element and we just return the value of the current node
        \end{itemize}
  \item Case 2: $k \leq leftSize$
        \begin{itemize}
          \item[--] This is the case when the solution is on the left subtree, so we recurse down the left child node of the root until the first case is met.
        \end{itemize}
  \item {Case 3: $k > leftSize$}
        \begin{itemize}
          \item[--] This is the case when the kth smallest element can be found in the right subtree and the algorithm recurses on the right. The recursive function call has the k being subtracted by \lstinline{leftSize} $- 1$ since we've already accounted for the nodes on the left subtree, if there is one.
        \end{itemize}
\end{itemize}
The next page shows the algorithm and the solutions obtained
\newpage
\begin{lstlisting}
  int select(Node root, int k) {
    // Find the median kth order statistic
    k = (k / 2) + 1;

    // Input validation
    if (k > root.size)
        return -1;

    // Prevents NullPointerExceptions
    int leftSize = root.left != null ? root.left.size : 0;
    
    if (k == leftSize + 1)
        return root.item;
     else if (k <= leftSize)
        return select(root.left, k);
    return select(root.right, k - leftSize - 1);
}
\end{lstlisting}
\begin{center}
  \large{Solutions obtained} \vspace{5mm} \\
  \begin{tblr}{hlines, vlines}
    Input                & k-th smallest element \\
    \{7, 10, 3, 13, 13\} & 10                    \\
    input-6.1.txt        & 501                   \\
    input-6.2.txt        & 5019                  \\
  \end{tblr}
\end{center}
\end{document}
