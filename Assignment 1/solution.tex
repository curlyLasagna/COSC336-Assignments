\documentclass[11pt]{article}
\usepackage{amsfonts}
\usepackage{amsmath}
\usepackage{latexsym}
\usepackage{hyperref}
\usepackage{pdfpages}
\usepackage{tikz}
\usepackage{fontspec}
\usepackage{fancybox}
\usepackage{listings}

\setlength{\oddsidemargin}{.25in}
\setlength{\evensidemargin}{.25in}
\setlength{\textwidth}{6in}
\setlength{\topmargin}{-0.4in}
\setlength{\textheight}{8.5in}

\def\squarebox#1{\hbox to #1{\hfill\vbox to #1{\vfill}}}
\def\qed{\hspace*{\fill}
        \vbox{\hrule\hbox{\vrule\squarebox{.667em}\vrule}\hrule}}
\newenvironment{solution}{\begin{trivlist}\item[]{\bf Solution:}}
                      {\qed \end{trivlist}}
\newenvironment{solsketch}{\begin{trivlist}\item[]{\bf Solution Sketch:}}
                      {\qed \end{trivlist}}
\newenvironment{proof}{\begin{trivlist}\item[]{\bf Proof:}}
                      {\qed \end{trivlist}}

\newtheorem{theorem}{Theorem}
\newtheorem{corollary}[theorem]{Corollary}
\newtheorem{lemma}[theorem]{Lemma}
\newtheorem{observation}[theorem]{Observation}
\newtheorem{remark}[theorem]{Remark}
\newtheorem{proposition}[theorem]{Proposition}
\newtheorem{definition}[theorem]{Definition}
\newtheorem{Assertion}[theorem]{Assertion}
\newtheorem{fact}[theorem]{Fact}
\newtheorem{hypothesis}[theorem]{Hypothesis}
%\newtheorem{observation}[theorem]{Observation}
%\newtheorem{proposition}[theorem]{Proposition}
\newtheorem{claim}[theorem]{Claim}
\newtheorem{assumption}[theorem]{Assumption}

%Put more macros here, as needed.
\newcommand{\al}{\alpha}
\newcommand{\Z}{\mathbb Z}
\newcommand{\jac}[2]{\left(\frac{#1}{#2}\right)}
\newcommand{\set}[1]{\{#1\}}
% Assignment header with the appropriate information
% 1st arg: Group member names
% 2nd arg: Assignment #
\newcommand{\header}[2]{
  \begin{center}
	\setlength\fboxsep{.3cm}
	\doublebox{
		\parbox{\textwidth} {
			#1 \\
			COSC 336 \\
			\today \par
			\centering{\huge{Assignment #2}}
		}}
\end{center}
}

\def\ppt{{\sf PPT}}
\def\poly{{\sf poly}}
\def\negl{{\sf negl}}
\def\owf{{\sf OWF}}
\def\owp{{\sf OWP}}
\def\tdp{{\sf TDP}}
\def\prg{{\sf PRG}}
\def\prf{{\sf PRF}}

\begin{document}

\setmainfont{Roboto}[
	BoldFont=*-Bold,
	UprightFont=*-Light,
]
\header{Luis Gascon, Ethan Webb, Femi Dosumu}{1}

\textbf{Exercise} \par
Consider the following three program fragments (a), (b), and (c).

% \lstset{
% 	mathescape,
% 	keywords={then, do, for, next, to, end, if},
% 	tabsize=2,
% 	autogobble=true
% }
\begin{itemize}
	\item[(a)]
		\begin{verbatim}
sum = 0;
for (int i = 0; i< n ; i++) {
    	   sum++;
}
\end{verbatim}

	\item[(b)]
		\begin{verbatim}
sum = 0;
for (int i = 0; i< 2*n ; i++) {
    	   sum++;
}
\end{verbatim}

	\item[(c)]
		\begin{verbatim}
sum = 0;  i=n*n;
while (i > 1) {
    	   sum++;
    	   i= i/2;
}
\end{verbatim}

We denote by $T_a(n), T_b(n), T_c(n)$ the running time of the three fragments.

\begin{enumerate}
\item  Give $\Theta$ evaluations for   $T_a(n), T_b(n), T_c(n)$. \\
\(T_{a}(n) = \Theta(n)\)
\item Is  $T_b(n)  = O(T_a(n))$ ? Answer YES or NO and justify your answer.
\item  Is $T_c(n) = \Theta (T_a(n))$ ?  Answer YES or NO and justify your answer.


\end{enumerate}


\end{itemize}
\end{document}
