\documentclass[11pt]{article}
\usepackage{amsfonts}
\usepackage{amsmath}
\usepackage{latexsym}
\usepackage{hyperref}
\usepackage{pdfpages}
\usepackage{tikz}
\usepackage{fontspec}
\usepackage{fancybox}
% \usepackage{fancyhdr}
% \pagestyle{fancy}

\setlength{\oddsidemargin}{.25in}
\setlength{\evensidemargin}{.25in}
\setlength{\textwidth}{6in}
\setlength{\topmargin}{-0.4in}
\setlength{\textheight}{8.5in}

\def\squarebox#1{\hbox to #1{\hfill\vbox to #1{\vfill}}}
\def\qed{\hspace*{\fill}
        \vbox{\hrule\hbox{\vrule\squarebox{.667em}\vrule}\hrule}}
\newenvironment{solution}{\begin{trivlist}\item[]{\bf Solution:}}
                      {\qed \end{trivlist}}
\newenvironment{solsketch}{\begin{trivlist}\item[]{\bf Solution Sketch:}}
                      {\qed \end{trivlist}}
\newenvironment{proof}{\begin{trivlist}\item[]{\bf Proof:}}
                      {\qed \end{trivlist}}

\newtheorem{theorem}{Theorem}
\newtheorem{corollary}[theorem]{Corollary}
\newtheorem{lemma}[theorem]{Lemma}
\newtheorem{observation}[theorem]{Observation}
\newtheorem{remark}[theorem]{Remark}
\newtheorem{proposition}[theorem]{Proposition}
\newtheorem{definition}[theorem]{Definition}
\newtheorem{Assertion}[theorem]{Assertion}
\newtheorem{fact}[theorem]{Fact}
\newtheorem{hypothesis}[theorem]{Hypothesis}
%\newtheorem{observation}[theorem]{Observation}
%\newtheorem{proposition}[theorem]{Proposition}
\newtheorem{claim}[theorem]{Claim}
\newtheorem{assumption}[theorem]{Assumption}

%Put more macros here, as needed.
\newcommand{\al}{\alpha}
\newcommand{\Z}{\mathbb Z}
\newcommand{\jac}[2]{\left(\frac{#1}{#2}\right)}
\newcommand{\set}[1]{\{#1\}}
% Assignment header with the appropriate information
% 1st arg: Group member names
% 2nd arg: Assignment #
\newcommand{\header}[2]{
  \begin{center}
	\setlength\fboxsep{.3cm}
	\doublebox{
		\parbox{\textwidth} {
			#1 \\
			COSC 336 \\
			\today \par
			\centering{\huge{Assignment #2}}
		}}
\end{center}
}

\def\ppt{{\sf PPT}}
\def\poly{{\sf poly}}
\def\negl{{\sf negl}}
\def\owf{{\sf OWF}}
\def\owp{{\sf OWP}}
\def\tdp{{\sf TDP}}
\def\prg{{\sf PRG}}
\def\prf{{\sf PRF}}

\begin{document}

\setmainfont{Roboto}[
	BoldFont=*-Bold,
    UprightFont=*-Light,
]
\header{Luis Gascon}{0}
\textbf{Example 1.} We define
\[S_{n}=1+2+\ldots+n\]
We present a proof of this formula without induction. We write $S_{n}$ in two ways as follows:
\begin{align*}
	S_{n} & = 1 + 2 + \ldots + (n-1) +n  \\
	S_{n} & = n + (n-1) + \ldots + 2 + 1
\end{align*}
Notice that on the right side we have two rows and $n$ columns. In each column the sum of the two numbers is \(n+1\). Indeed, the sum in the first column is \(1 + n = n +1\), in the second column is \(2+(n-1) = n+1\), and so on. \\
So, if add the two rows we obtain
\[2S_{n}=(n+1)+(n+1)+\ldots+(n+1) = n \times (n+1)\]
and therefore \(S_{n}=n(n+1)/2\).
\vspace*{.5cm} \\
Of course, there is also a proof by induction, but it is less fun.
\vspace*{.5cm} \\
\textbf{Example 2.} The following is a directed graph with 3 vertices and 3 edges:
\vspace*{-3mm}
\begin{figure}[htp]
	\centering
	\begin{tikzpicture}
		% Draw the vertices with the labels up top
		\filldraw[black] (0, 0) circle(2pt) node[anchor=south, inner sep=2mm]{a};
		\filldraw[black] (1.6, 2) circle(2pt) node[anchor=south, inner sep=1.5mm]{c};
		\filldraw[black] (3.2, 0) circle(2pt) node[anchor=south, inner sep=2mm]{b};
		\filldraw[black] (7.04, 0) circle(2pt) node[anchor=south, inner sep=1.5mm]{d};

		% Draw edges
		\draw[black, thick] (0, 0) -- (1.6, 2);
		\draw[black, thick] (1.6, 2) -- (3.2, 0);
		\draw[black, thick] (0, 0) -- (3.2, 0);
		\draw[black, thick] (3.2, 0) -- (7.04, 0);
	\end{tikzpicture}
\end{figure} \\
\textbf{Example 3.} Here is formula involving the greek letters \(\alpha, \beta \text{ and } \epsilon \):
\[\alpha^{2}+\beta^{2}=\epsilon^{2}.\]
\end{document}
